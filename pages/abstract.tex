\chapter{\abstractname}

This thesis explores the integration of \ac{GPR} into portfolio optimization, presenting a novel framework that leverages predictive modeling to enhance investment performance. \ac{GPR}, a non-parametric Bayesian approach, is employed to forecast asset returns and associated uncertainties, addressing key challenges in financial forecasting such as non-linearity, non-stationarity, and data noise. The study develops a Dynamic Strategy that incorporates these predictions into portfolio allocation, allowing for adaptive decision-making based on probabilistic thresholds.

The methodology includes comprehensive data preprocessing, model development, and backtesting over historical market data, encompassing diverse assets across major sectors. Four portfolio strategies—Maximum Return, Minimum Volatility, Maximum Sharpe Ratio, and the proposed Dynamic Strategy—are evaluated on performance metrics including total returns, volatility, Sharpe ratios, and transaction costs. The results demonstrate that the Dynamic Strategy consistently outperforms traditional approaches, achieving higher risk-adjusted returns while minimizing transaction costs through intelligent rebalancing.

This research contributes to the field of predictive portfolio optimization by introducing a robust framework that integrates advanced forecasting with adaptive asset allocation. Practical implications for portfolio managers include improved risk management, reduced rebalancing costs, and enhanced decision-making in dynamic markets. Additionally, the thesis identifies opportunities for future research, such as refining GPR models, expanding strategy applications, and exploring scalability for large portfolios. These findings highlight the transformative potential of machine learning in modern financial decision-making.