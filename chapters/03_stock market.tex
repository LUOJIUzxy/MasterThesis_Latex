% !TeX root = ../main.tex
% Add the above to each chapter to make compiling the PDF easier in some editors.

\chapter{The US Stock Market}\label{chapter:stock market}

\section{Section}
Citation test~\parencite{latex}.

Acronyms must be added in \texttt{main.tex} and are referenced using macros. The first occurrence is automatically replaced with the long version of the acronym, while all subsequent usages use the abbreviation.

E.g. \texttt{\textbackslash ac\{TUM\}, \textbackslash ac\{TUM\}} $\Rightarrow$ \ac{TUM}, \ac{TUM}

For more details, see the documentation of the \texttt{acronym} package\footnote{\url{https://ctan.org/pkg/acronym}}.
\subsection{Subsection}

See~\autoref{tab:sample}, \autoref{fig:sample-drawing}, \autoref{fig:sample-plot}, \autoref{fig:sample-listing}.

The idea combines not only return and volatility but also the fixed-income market. The fixed-income market is important because it shows what any risk-adverse investor would earn by simply being in the most conservative investment: the risk-free rate. The risk-free rate is the rate earned simply for investing in a risk-free bond. Since stocks are risky, they should earn returns in excess of that return. If a stock were to earn less than the risk-free return, then a rational investor would forgo the volatility of the stock market and stick with risk-free investments, which protect the principal while still earning some return.

The numerator in our example is the market premium. So, suppose for stock C, the market premium and standard deviation are 7\% and 10\%, respectively. Suppose for stock D, the market premium and standard deviation are 8\% and 12\%, respectively. Which is the preferred investment?
In this case, stock C has a higher return per unit of volatility (risk) than stock D does. We would prefer the investment that has a higher amount of return per unit of risk. Now, this is the opposite of our coefficient of variation. When we express volatility units per return, we would like the lower number. When we express return units per volatility, we prefer the larger number. Here, the larger number means we expect to earn more return per unit of volatility.

In this case, we will name this ratio (keep a *sharp* lookout for it) and use it extensively to compare investments, but for now, the important thing is that we have seen how return and volatility can be combined to compare investments. This can just as easily work if these securities are from different asset classes. Return and risk help encapsulate the key statistical properties of a financial asset's performance. In the next lesson, we’ll dive deeper into the details of these statistics.
\begin{table}[htpb]
  \caption[Example table]{An example for a simple table.}\label{tab:sample}
  \centering
  \begin{tabular}{l l l l}
    \toprule
      A & B & C & D \\
    \midrule
      1 & 2 & 1 & 2 \\
      2 & 3 & 2 & 3 \\
    \bottomrule
  \end{tabular}
\end{table}

\begin{figure}[htpb]
  \centering
  % This should probably go into a file in figures/
  \begin{tikzpicture}[node distance=3cm]
    \node (R0) {$R_1$};
    \node (R1) [right of=R0] {$R_2$};
    \node (R2) [below of=R1] {$R_4$};
    \node (R3) [below of=R0] {$R_3$};
    \node (R4) [right of=R1] {$R_5$};

    \path[every node]
      (R0) edge (R1)
      (R0) edge (R3)
      (R3) edge (R2)
      (R2) edge (R1)
      (R1) edge (R4);
  \end{tikzpicture}
  \caption[Example drawing]{An example for a simple drawing.}\label{fig:sample-drawing}
\end{figure}

\begin{figure}[htpb]
  \centering

  \pgfplotstableset{col sep=&, row sep=\\}
  % This should probably go into a file in data/
  \pgfplotstableread{
    a & b    \\
    1 & 1000 \\
    2 & 1500 \\
    3 & 1600 \\
  }\exampleA
  \pgfplotstableread{
    a & b    \\
    1 & 1200 \\
    2 & 800 \\
    3 & 1400 \\
  }\exampleB
  % This should probably go into a file in figures/
  \begin{tikzpicture}
    \begin{axis}[
        ymin=0,
        legend style={legend pos=south east},
        grid,
        thick,
        ylabel=Y,
        xlabel=X
      ]
      \addplot table[x=a, y=b]{\exampleA};
      \addlegendentry{Example A};
      \addplot table[x=a, y=b]{\exampleB};
      \addlegendentry{Example B};
    \end{axis}
  \end{tikzpicture}
  \caption[Example plot]{An example for a simple plot.}\label{fig:sample-plot}
\end{figure}

\begin{figure}[htpb]
  \centering
  \begin{tabular}{c}
  \begin{lstlisting}[language=SQL]
    SELECT * FROM tbl WHERE tbl.str = "str"
  \end{lstlisting}
  \end{tabular}
  \caption[Example listing]{An example for a source code listing.}\label{fig:sample-listing}
\end{figure}
