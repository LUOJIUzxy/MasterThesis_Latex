% !TeX root = ../main.tex
% Add the above to each chapter to make compiling the PDF easier in some editors.

\chapter{Introduction}\label{chapter:introduction}

\section{Background and Motivation}
Citation test~\parencite{latex}.

Acronyms must be added in \texttt{main.tex} and are referenced using macros. The first occurrence is automatically replaced with the long version of the acronym, while all subsequent usages use the abbreviation.

E.g. \texttt{\textbackslash ac\{TUM\}, \textbackslash ac\{TUM\}} $\Rightarrow$ \ac{TUM}, \ac{TUM}

For more details, see the documentation of the \texttt{acronym} package\footnote{\url{https://ctan.org/pkg/acronym}}.


\subsection{Evolution of portfolio optimization techniques}
The field of portfolio optimization has undergone significant transformations since its inception in the mid-20th century, evolving from simple diversification principles to sophisticated mathematical models incorporating machine learning and artificial intelligence. 
This evolution reflects both the advancing computational capabilities and our deepening understanding of financial markets' complexity.


\paragraph{Classical Foundations (1950s-1960s)}
The modern era of portfolio optimization began with \parencite{markowitz1952portfolio}'s seminal paper ``Portfolio Selection,'' which laid the groundwork for \ac{MPT}. Markowitz introduced several revolutionary concepts:

\begin{itemize}
    \item The mathematical formalization of diversification
    \item The mean-variance optimization framework
    \item The efficient frontier of optimal portfolios
    \item The fundamental relationship between risk and return
\end{itemize}

This work established the first rigorous mathematical framework for portfolio selection, earning Markowitz the Nobel Prize in Economics and fundamentally changing how practitioners approached portfolio management.

\paragraph{Early Developments (1960s-1980s)}
Building upon Markowitz's foundation, several crucial developments emerged:

\begin{enumerate}
    \item \textbf{Capital Asset Pricing Model (CAPM)}
    \begin{itemize}
        \item Developed by \parencite{sharpe1964capital}, \parencite{lintner1965security}, and \parencite{mossin1966equilibrium}
        \item Introduced systematic and unsystematic risk concepts
        \item Established the theoretical framework for asset pricing
        \item Created the foundation for risk-adjusted performance measures
    \end{itemize}

    \item \textbf{Single-Index Models}
    \begin{itemize}
        \item Simplified the estimation of covariance matrices
        \item Reduced computational complexity
        \item Introduced market beta as a risk measure
        \item Enhanced practical applicability of portfolio optimization
    \end{itemize}
\end{enumerate}

\paragraph{Advanced Optimization Era (1980s-2000s)}
The advent of increased computational power led to more sophisticated approaches:

\begin{enumerate}
    \item \textbf{Black-Litterman Model (1990s)}
    \begin{itemize}
        \item Incorporated investor views into the optimization process
        \item Addressed estimation error issues in mean-variance optimization
        \item Introduced Bayesian methods to portfolio optimization
        \item Provided more stable and intuitive portfolio allocations
    \end{itemize}

    \item \textbf{Risk-Based Portfolio Optimization}
    \begin{itemize}
        \item Development of risk parity strategies \parencite{qian2005risk}
        \item Introduction of alternative risk measures (VaR, CVaR)
        \item Focus on downside risk management
        \item Enhanced robustness to estimation errors
    \end{itemize}
\end{enumerate}

\paragraph{Modern Approaches (2000s-Present)}
Recent developments have focused on addressing classical methods' limitations:

\begin{enumerate}
    \item \textbf{Robust Optimization}
    \begin{itemize}
        \item Accounts for parameter uncertainty
        \item Provides protection against worst-case scenarios
        \item Incorporates estimation error in the optimization process
        \item Yields more stable portfolio allocations
    \end{itemize}

    \item \textbf{Dynamic Portfolio Optimization}
    \begin{itemize}
        \item Considers time-varying investment opportunities
        \item Incorporates transaction costs
        \item Accounts for changing market conditions
        \item Enables adaptive portfolio management
    \end{itemize}

    \item \textbf{Machine Learning Integration}
    \begin{itemize}
        \item Neural networks for return prediction \parencite{heaton2017deep}
        \item Support vector machines for risk assessment
        \item Reinforcement learning for portfolio management
        \item \ac{GP} for uncertainty quantification
    \end{itemize}
\end{enumerate}

\paragraph{Current Challenges and Future Directions}
Modern portfolio optimization faces several challenges:

\begin{enumerate}
    \item \textbf{Data Quality and Quantity}
    \begin{itemize}
        \item High-dimensional data processing
        \item Non-stationary market conditions
        \item Alternative data integration
        \item Real-time data processing requirements
    \end{itemize}

    \item \textbf{Model Complexity}
    \begin{itemize}
        \item Balance between model sophistication and robustness
        \item Computational efficiency
        \item Interpretability of results
        \item Parameter stability
    \end{itemize}

    \item \textbf{Implementation Challenges}
    \begin{itemize}
        \item Transaction costs
        \item Market impact
        \item Regulatory constraints
        \item Operational considerations
    \end{itemize}
\end{enumerate}

The field continues to evolve with emerging technologies and methodologies:

\begin{enumerate}
    \item \textbf{Artificial Intelligence Applications}
    \begin{itemize}
        \item Deep learning for market prediction
        \item Natural language processing for sentiment analysis
        \item Alternative data processing
        \item Automated portfolio rebalancing
    \end{itemize}

    \item \textbf{Advanced Risk Management}
    \begin{itemize}
        \item Tail risk hedging
        \item Dynamic risk allocation
        \item Scenario analysis
        \item Real-time risk monitoring
    \end{itemize}

    \item \textbf{Sustainability Integration}
    \begin{itemize}
        \item ESG factors in optimization
        \item Climate risk consideration
        \item Impact investing metrics
        \item Sustainable portfolio construction
    \end{itemize}
\end{enumerate}

This evolution of portfolio optimization techniques sets the stage for our research, which builds upon these foundations while incorporating modern machine learning approaches, specifically Gaussian Process Regression, to address current challenges in portfolio optimization.


\subsection{Role of machine learning in financial forecasting}
\ac{ML} has gained significant traction in financial markets due to its ability to analyze vast amounts of data and identify complex patterns that traditional models may overlook.
In particular, \ac{GPR} has emerged as a powerful tool for time-series forecasting, offering a flexible framework for capturing non-linear relationships and uncertainty in predictions.

\subsection{Challenges in time-series forecasting and traditional portfolio optimization methods}
Despite the advancements in \ac{ML} techniques, predicting asset returns remains a challenging task due to the inherent volatility and non-stationarity of financial markets.
Moreover, traditional portfolio optimization methods, while theoretically elegant, often face significant practical challenges in implementation. These challenges primarily stem from the difficulty in accurately estimating input parameters and the inherent uncertainty in financial time-series forecasting. 
This section examines these challenges and introduces how \ac{GPR} provides a novel approach to addressing them.

\paragraph{Parameter Estimation Challenges in Modern Portfolio Theory}
Modern Portfolio Theory (MPT), despite its theoretical elegance, relies heavily on accurate estimation of key parameters:
The practical implementation of MPT is fundamentally constrained by the difficulty in estimating volatility, arguably the most critical parameter in portfolio optimization. Traditional approaches to volatility estimation rely heavily on historical data, assuming that past patterns will persist into the future. However, financial markets are dynamic systems characterized by regime changes, varying volatility clusters, and complex interdependencies that make such assumptions problematic. Historical volatility estimates are inherently backward-looking and highly sensitive to the chosen estimation window, leading to potentially misleading inputs for portfolio optimization.

Moreover, the challenge extends beyond simple volatility estimation. The correlation structure between assets, another crucial input for MPT, exhibits time-varying properties that are difficult to capture using conventional methods. During periods of market stress, these correlations often shift dramatically, invalidating historical estimates precisely when accurate risk assessment is most critical. The dimensionality of this problem grows quadratically with the number of assets, making it particularly challenging for large, diversified portfolios.

\begin{itemize}
    \item \textbf{Volatility Estimation}
    \begin{itemize}
        \item Historical volatility may not reflect future risk
        \item Sample estimates are sensitive to the chosen time window
        \item Regime changes can invalidate historical estimates
        \item Heteroskedasticity in financial time series complicates estimation
    \end{itemize}

    \item \textbf{Expected Returns}
    \begin{itemize}
        \item Notoriously difficult to estimate accurately
        \item High sensitivity to estimation errors
        \item Time-varying nature of expected returns
        \item Impact of market regimes on return distributions
    \end{itemize}

    \item \textbf{Correlation Structure}
    \begin{itemize}
        \item Dynamic nature of asset correlations
        \item Curse of dimensionality in large portfolios
        \item Instability during market stress periods
        \item Computational challenges in high dimensions
    \end{itemize}
\end{itemize}

\paragraph{Limitations of Traditional Forecasting Methods}
Traditional forecasting approaches in finance have predominantly relied on methods that provide point estimates, failing to capture the inherent uncertainty in financial predictions. These methods often make strong assumptions about the underlying data distribution and struggle to adapt to the non-linear, non-stationary nature of financial time series. \ac{ARIMA} models, exponential smoothing, and other classical approaches, while mathematically tractable, often fall short in capturing the complex dynamics of financial markets.

A fundamental limitation of these traditional approaches is their rigidity in handling uncertainty. Point forecasts, even when accompanied by confidence intervals based on historical volatility, fail to capture the dynamic nature of prediction uncertainty. This limitation becomes particularly problematic in portfolio optimization, where understanding the reliability of forecasts is as important as the forecasts themselves.

Conventional approaches to financial time-series forecasting exhibit several limitations:

\begin{enumerate}
    \item \textbf{Point Estimates}
    \begin{itemize}
        \item Traditional methods often provide single-point forecasts
        \item Lack of uncertainty quantification
        \item Limited ability to capture prediction confidence
        \item Insufficient information for risk management
    \end{itemize}

    \item \textbf{Model Rigidity}
    \begin{itemize}
        \item Assumption of specific probability distributions
        \item Difficulty in capturing non-linear relationships
        \item Limited adaptation to changing market conditions
        \item Oversimplification of complex market dynamics
    \end{itemize}

    \item \textbf{Data Requirements}
    \begin{itemize}
        \item Need for large historical datasets
        \item Sensitivity to outliers and noise
        \item Challenge of incorporating multiple data sources
        \item Difficulty in handling missing data
    \end{itemize}
\end{enumerate}

\paragraph{Advantages of Gaussian Process Regression}
Our research proposes \ac{GPR} as a solution to these challenges, offering several key advantages:

\begin{enumerate}
    \item \textbf{Probabilistic Framework}
    \begin{itemize}
        \item Natural uncertainty quantification
        \item Automatic volatility estimation through posterior variance
        \item Capture of prediction confidence intervals
        \item Robust handling of noise in financial data
    \end{itemize}

    \item \textbf{Flexible Modeling}
    \begin{itemize}
        \item Non-parametric approach avoiding distributional assumptions
        \item Ability to capture complex non-linear relationships
        \item Automatic complexity adjustment through kernel selection
        \item Incorporation of prior knowledge through kernel design
    \end{itemize}

    \item \textbf{Parameter Estimation}
    \begin{itemize}
        \item Direct modeling of volatility through posterior variance
        \item Joint estimation of returns and risk
        \item Principled handling of uncertainty
        \item Adaptive to changing market conditions
    \end{itemize}
\end{enumerate}

\paragraph{Addressing Traditional Limitations}
Our \ac{GPR}-based approach specifically addresses the key limitations of \ac{MPT}:

\begin{equation}
    \sigma_{GPR}^2(x_*) = k(x_*, x_*) - k(x_*, X)[K(X,X) + \sigma_n^2I]^{-1}k(X, x_*)
    \label{eq:gpr_variance}
\end{equation}

Where $\sigma_{GPR}^2(x_*)$ represents the posterior variance at prediction point $x_*$, providing a direct estimate of volatility that:

\begin{itemize}
    \item Naturally accounts for uncertainty in predictions
    \item Adapts to local data density and quality
    \item Provides time-varying volatility estimates
    \item Incorporates both local and global market information
\end{itemize}

The \ac{GPR} approach offers several fundamental advantages over traditional methods. Unlike historical volatility estimates that require arbitrary window selection, \ac{GPR}'s volatility estimates emerge naturally from the probabilistic learning process. The method adapts automatically to different market regimes through its kernel function, which can capture both long-term trends and short-term fluctuations in market behavior.

Furthermore, \ac{GPR}'s non-parametric nature frees it from restrictive assumptions about return distributions. The method can capture complex, non-linear patterns in the data while maintaining the ability to quantify uncertainty in its predictions. This combination of flexibility and uncertainty awareness makes it particularly well-suited for financial applications where both accuracy and risk assessment are crucial.


\paragraph{Implications for Portfolio Optimization}
The \ac{GPR} framework transforms the traditional portfolio optimization problem by:
The integration of \ac{GPR} into portfolio optimization transforms the traditional \ac{MPT} framework by providing more reliable and dynamic parameter estimates. By directly modeling the uncertainty in our predictions, we can make more informed portfolio allocation decisions that account for both expected returns and our confidence in those expectations. This approach naturally leads to more robust portfolios that adapt to changing market conditions while maintaining a principled approach to risk management.

The significance of this advancement cannot be overstated. By addressing one of the fundamental criticisms of \ac{MPT} – the difficulty of accurately estimating volatility – our \ac{GPR}-based approach bridges the gap between theoretical elegance and practical applicability. This enhancement makes \ac{MPT} more reliable and useful for real-world portfolio management, where accurate risk assessment is crucial for maintaining stable, long-term investment performance.


\section{Research Objectives}
The primary objective of this study is to develop a predictive portfolio optimization framework that leverages Gaussian Process Regression (GPR) for time-series forecasting in financial markets. 
By integrating advanced predictive modeling with strategic asset allocation, the study aims to enhance investment performance through informed decision-making. 
The specific objectives are as follows:

\begin{enumerate}
    \item \textbf{Develop and Validate \ac{GPR} Models for Asset Return Prediction}
    \begin{itemize}
        \item \emph{Model Construction:} Build individual GPR models to forecast future returns of selected assets, including forex, gold, Bitcoin, and various stocks.
        \item \emph{Feature Engineering:} Utilize historical one-month returns and time as input features to capture both market dynamics and temporal patterns.
        \item \emph{Model Updating:} Implement an iterative training process where models are updated daily with new market data to ensure predictions remain current and adaptive.
    \end{itemize}
    
    \item \textbf{Integrate \ac{GPR} Predictions into Portfolio Optimization Strategies}
    \begin{itemize}
        \item \emph{Expected Returns and Volatilities:} Extract predicted returns and associated volatilities from the GPR models for use in portfolio construction.
        \item \emph{Strategy Formulation:} Design multiple optimization strategies---Maximum Return, Minimum Volatility, Maximum Sharpe Ratio, and a Dynamic Strategy---based on the \ac{GPR} outputs.
    \end{itemize}
    
    \item \textbf{Develop a Dynamic Portfolio Optimization Strategy}
    \begin{itemize}
        \item \emph{Probabilistic Assessment:} Calculate the probability distribution of next-day cumulative portfolio returns using the predicted normal distribution of asset returns.
        \item \emph{Threshold-Based Decision Making:} Establish a threshold probability to decide when to reallocate the portfolio for maximizing returns versus holding the current positions.
        \item \emph{Adaptive Allocation:} Enable the portfolio to adapt dynamically to changing market conditions by selectively applying the Maximum Return Strategy based on probabilistic forecasts.
    \end{itemize}
    
    \item \textbf{Evaluate and Compare the Performance of Optimization Strategies}
    \begin{itemize}
        \item \emph{Backtesting Framework:} Conduct backtesting over a historical period using real market data to assess the strategies' performance.
        \item \emph{Incorporation of Transaction Costs:} Include realistic transaction fees in the evaluation to account for the costs associated with portfolio rebalancing.
        \item \emph{Performance Metrics:} Measure total returns, portfolio volatility, Sharpe ratios, and transaction costs to provide a comprehensive performance analysis.
    \end{itemize}
    
    \item \textbf{Demonstrate the Effectiveness of the Dynamic Strategy}
    \begin{itemize}
        \item \emph{Performance Analysis:} Analyze the results to determine if the Dynamic Strategy achieves higher returns and lower transaction costs compared to traditional strategies.
        \item \emph{Risk Management:} Assess how the Dynamic Strategy balances the trade-off between pursuing higher returns and minimizing risks and costs.
        \item \emph{Statistical Significance:} Use statistical methods to verify the significance of the observed performance differences among the strategies.
    \end{itemize}
    
    \item \textbf{Contribute to the Field of Predictive Portfolio Optimization}
    \begin{itemize}
        \item \emph{Innovative Approach:} Present a novel integration of GPR-based forecasting with adaptive portfolio optimization strategies.
        \item \emph{Practical Implications:} Provide insights and recommendations for practitioners on implementing dynamic, data-driven approaches in portfolio management.
        \item \emph{Foundation for Future Research:} Establish a basis for further exploration into advanced predictive models and adaptive strategies in financial optimization.
    \end{itemize}
\end{enumerate}

By achieving these objectives, the study seeks to demonstrate that incorporating sophisticated predictive models like \ac{GPR} into portfolio optimization can significantly enhance investment outcomes. The findings aim to contribute valuable knowledge to the field of quantitative finance, particularly in the areas of time-series forecasting and dynamic asset allocation.

\subsection{Research Contributions}

This study makes several significant contributions to the field of predictive portfolio optimization and quantitative finance. The key research contributions are outlined below:

\begin{enumerate}
    \item \textbf{Novel Integration of Gaussian Process Regression with Dynamic Portfolio Optimization}

    This research presents a unique integration of \ac{GPR} models with dynamic portfolio optimization strategies. By employing \ac{GPR} for time-series forecasting, we capture complex, non-linear relationships in financial data, enhancing the accuracy of return predictions. The integration facilitates a more responsive and informed portfolio allocation process, adapting to market changes in real-time.

    \item \textbf{Probabilistic Approach to Strategy Selection}

    We introduce a probabilistic framework for strategy selection within the portfolio optimization process. By calculating the probability distribution of future cumulative returns, the Dynamic Strategy makes informed decisions on whether to reallocate the portfolio based on a predefined threshold. This approach incorporates uncertainty and risk directly into the decision-making process, allowing for a more nuanced and adaptive investment strategy.

    \item \textbf{Practical Implementation Considering Transaction Costs}

    The study emphasizes practical applicability by incorporating realistic transaction costs into the optimization and backtesting processes. By accounting for these costs, we provide a more accurate assessment of the strategies' net performance. This consideration is crucial for real-world portfolio management, where transaction fees can significantly impact returns, especially in high-frequency trading environments.

    \item \textbf{Comparative Analysis of Different Optimization Strategies}

    We conduct a comprehensive comparative analysis of multiple portfolio optimization strategies, including Maximum Return, Minimum Volatility, Maximum Sharpe Ratio, and the proposed Dynamic Strategy. By evaluating these strategies under the same conditions and performance metrics, we provide valuable insights into their relative effectiveness. This analysis helps identify the strengths and limitations of each approach, guiding practitioners in selecting appropriate strategies based on their investment goals and risk tolerance.

\end{enumerate}

These contributions collectively advance the understanding of how advanced predictive models and adaptive strategies can be effectively combined to enhance portfolio performance. The novel methodologies and findings offer practical benefits for portfolio managers and lay the groundwork for future research in predictive asset allocation.




\begin{table}[htpb]
  \caption[Example table]{An example for a simple table.}
  \centering
  \begin{tabular}{l l l l}
    \toprule
      A & B & C & D \\
    \midrule
      1 & 2 & 1 & 2 \\
      2 & 3 & 2 & 3 \\
    \bottomrule
  \end{tabular}
\end{table}

\begin{figure}[htpb]
  \centering
  % This should probably go into a file in figures/
  \begin{tikzpicture}[node distance=3cm]
    \node (R0) {$R_1$};
    \node (R1) [right of=R0] {$R_2$};
    \node (R2) [below of=R1] {$R_4$};
    \node (R3) [below of=R0] {$R_3$};
    \node (R4) [right of=R1] {$R_5$};

    \path[every node]
      (R0) edge (R1)
      (R0) edge (R3)
      (R3) edge (R2)
      (R2) edge (R1)
      (R1) edge (R4);
  \end{tikzpicture}
  \caption[Example drawing]{An example for a simple drawing.}
\end{figure}

\begin{figure}[htpb]
  \centering

  \pgfplotstableset{col sep=&, row sep=\\}
  % This should probably go into a file in data/
  \pgfplotstableread{
    a & b    \\
    1 & 1000 \\
    2 & 1500 \\
    3 & 1600 \\
  }\exampleA
  \pgfplotstableread{
    a & b    \\
    1 & 1200 \\
    2 & 800 \\
    3 & 1400 \\
  }\exampleB
  % This should probably go into a file in figures/
  \begin{tikzpicture}
    \begin{axis}[
        ymin=0,
        legend style={legend pos=south east},
        grid,
        thick,
        ylabel=Y,
        xlabel=X
      ]
      \addplot table[x=a, y=b]{\exampleA};
      \addlegendentry{Example A};
      \addplot table[x=a, y=b]{\exampleB};
      \addlegendentry{Example B};
    \end{axis}
  \end{tikzpicture}
  \caption[Example plot]{An example for a simple plot.}
\end{figure}

\begin{figure}[htpb]
  \centering
  \begin{tabular}{c}
  \begin{lstlisting}[language=SQL]
    SELECT * FROM tbl WHERE tbl.str = "str"
  \end{lstlisting}
  \end{tabular}
  \caption[Example listing]{An example for a source code listing.}
\end{figure}
